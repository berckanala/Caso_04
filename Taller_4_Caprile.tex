\documentclass[11pt]{article}  % Define la clase del documento.

% Paquetes de idioma y codificación
\usepackage[utf8]{inputenc}
\usepackage[T1]{fontenc}
\usepackage{tabularx}

\usepackage{caption}
\usepackage{subcaption}

% Geometría
\usepackage[letterpaper, margin=3cm]{geometry}

\usepackage[spanish]{babel}
\addto\captionsspanish{%
  \renewcommand{\tablename}{Tabla}%
  \renewcommand{\figurename}{Figura}%
}

% Tipografía
\usepackage{mathptmx}
\usepackage{microtype}

% Gráficos y colores
\usepackage{xcolor}
\usepackage{graphicx}
\usepackage{tikz}
\usepackage{multirow}

% Hipervínculos
\usepackage{hyperref}
\definecolor{darkblue}{rgb}{0.0, 0.0, 0.55}
\hypersetup{
    colorlinks   = true,
    linkcolor    = darkblue,
    citecolor    = black,
    filecolor    = blue,
    urlcolor     = blue
}

\usepackage{media9}

% Tablas y flotantes
\usepackage{booktabs}
\usepackage{float}

% Listados (opcional)
\usepackage{listings}
\definecolor{codegreen}{rgb}{0.25, 0.49, 0.48}
\definecolor{codegray}{rgb}{0.5, 0.5, 0.5}
\definecolor{codepurple}{rgb}{0.58, 0, 0.82}
\definecolor{backcolour}{rgb}{0.95, 0.95, 0.92}
\lstset{
    literate=
    {á}{{\'a}}1 {é}{{\'e}}1 {í}{{\'i}}1 {ó}{{\'o}}1 {ú}{{\'u}}1
    {Á}{{\'A}}1 {É}{{\'E}}1 {Í}{{\'I}}1 {Ó}{{\'O}}1 {Ú}{{\'U}}1
    {ñ}{{\~n}}1 {Ñ}{{\~N}}1 {ü}{{\"u}}1 {Ü}{{\"U}}1,
    backgroundcolor=\color{backcolour},
    commentstyle=\color{codegreen},
    keywordstyle=\color{codepurple},
    numberstyle=\tiny\color{codegray},
    stringstyle=\color{red},
    basicstyle=\ttfamily\small,
    breaklines=true,
    captionpos=b,
    numbers=left,
    numbersep=5pt,
    showstringspaces=false,
    tabsize=2,
    language=TeX,
    frame=single,
    rulecolor=\color{black}
}

% Párrafo y matemáticas
\usepackage{amsmath}
\usepackage{parskip}
\usepackage{ragged2e}
\usepackage{multicol}

% Encabezados y pies
\usepackage{titlesec}
\titleclass{\part}{top}
\titleformat{\part}[display]
  {\normalfont\huge\centering}{\thepart}{40pt}{\Huge}
\titlespacing*{\part}{0pt}{-60pt}{10pt}
\titleformat{\part}
  {\normalfont\huge}{}{0pt}{}

\usepackage{fancyhdr}
\pagestyle{fancy}
\fancyhf{}
\fancyhead[L]{\raisebox{0.20cm}{Tecnologías del Hormigón}}
\fancyhead[R]{\raisebox{0.1cm}{\includegraphics[width=0.25\linewidth]{LOGO_UNIVERSIDAD.jpg}}}
\fancyhead[C]{\rule{\textwidth}{0.6pt}}
\fancyfoot[C]{\rule{\textwidth}{0.6pt}}
\fancyfoot[R]{\raisebox{-1.5\baselineskip}{\thepage}}
\renewcommand{\headrulewidth}{0pt}
\renewcommand{\footrulewidth}{0pt}

\geometry{
  top=3.5cm,
  bottom=2.5cm,
  headheight=2.5cm
}

% Bibliografía
\usepackage{natbib}
\bibliographystyle{unsrtnat}

\begin{document}

%----------------------------------------------------------------------------------------
% PORTADA
%----------------------------------------------------------------------------------------
\begin{titlepage}
\newcommand{\HRule}{\rule{\linewidth}{0.5mm}}
\centering
\includegraphics[width=10cm]{LOGO_UNIVERSIDAD.jpg}\\[3cm]
\HRule \\[0.4cm]
{\huge Tecnologías del Hormigón}\\[0.4cm]
{\huge Taller 4: Hormigón Autocompactante - Segunda parte}\\[0.4cm]
\HRule \\[1.5cm]
\vspace{5cm}
\begin{flushright}
Profesor:\\
Álvaro Paul\\
Alumnos:\\
Bernardo Caprile\\
Pedro Valenzuela\\
\end{flushright}
\vspace{1cm}
{\large \today}\\[2cm]
\end{titlepage}

\newpage
\thispagestyle{empty}

%----------------------------------------------------------------------------------------
% INICIO DEL INFORME
%----------------------------------------------------------------------------------------
\setcounter{page}{1}

\section{Desarrollo}
\subsection*{Pregunta 1}
\subsubsection*{Inciso a)}
De la tabla 1 del enunciado se pueden obtener los siguientes parámetros importantes:
\begin{table}[h]
\centering
\caption{Parámetros importantes de las dos mezclas de hormigón.}
  \begin{tabular}{ccc}
  \hline
  \textbf{Parámetro} & \textbf{Hormigón 1} & \textbf{Hormigón 2}  \\ \hline
  w/c & 0.4 & 0.51  \\ \hline
  Aridos finos & 53 \% & 42\%  \\ \hline
  Aditivo reductor de agua & 1.25\% & 0.45\% \\ \hline
  \end{tabular}
\end{table}

Luego ocupando la tabla 4.2 del ACI 237R-07 [1] se pueden obtener los siguientes rangos para cada parámetro:
\begin{figure}[h]
  \centering
  \includegraphics[width=0.6\linewidth]{Tabla4.2.1.png}
  \caption{Rangos recomendados para los parámetros del hormigón autocompactante según ACI 237R-07 [1].}
  \label{tabla_scc}
\end{figure}

Con la tabla \ref{tabla_scc}, se puede observar que el hormigón 1 cumple con todos los parámetros recomendados para ser considerado un hormigón autocompactante, mientras que el hormigón 2 no cumple con el contenido de áridos finos ni con la cantidad de aditivo reductor de agua. Además, el hormigón 2 tiene una cantidad de aditivo reductor de agua muy baja, lo que puede afectar negativamente sus propiedades de fluidez y cohesión.


\subsubsection*{b) Propiedades en estado fresco}

Las diferencias entre ambas mezclas afectan directamente su comportamiento en estado fresco. El Hormigón 1 tiene más cemento, más finos y mayor dosis de aditivo reductor, lo que genera una pasta más abundante y fluida. Esto mejora su trabajabilidad, permitiendo que el hormigón fluya y llene moldajes con facilidad, e incluso favorece la autocompactación sin necesidad de vibración intensa. Además, el mayor contenido de finos le entrega cohesión, reduciendo la segregación y manteniendo la mezcla estable durante el vertido.

En cambio, el Hormigón 2 contiene más árido grueso, menos pasta y una menor dosificación de aditivo. Esto produce una mezcla más rígida y menos fluida, que depende fuertemente de la vibración para acomodarse. Su menor cohesión aumenta el riesgo de segregación o exudación si no se controla bien el proceso de colocación.

\subsubsection*{c) Propiedades en estado endurecido}

En el estado endurecido, el Hormigón 1 presenta mejores condiciones para obtener mayor resistencia y durabilidad, principalmente por su menor relación agua/cemento y su capacidad de compactarse de forma más uniforme. Esto genera una pasta más densa, con menos poros y mejor protección frente a agentes agresivos. 

Por su parte, el Hormigón 2, al tener una relación agua/cemento más alta y depender de la vibración, tiende a presentar una resistencia menor y una durabilidad reducida si no se compacta correctamente. El mayor contenido de árido grueso puede aumentar su módulo de elasticidad en comparación con una mezcla más rica en pasta, pero la presencia de vacíos o mala compactación puede contrarrestar este beneficio. En términos de densidad, el Hormigón 2 debería ser más pesado por su mayor contenido de árido, aunque esto depende de la calidad de la compactación alcanzada en obra.


\subsection*{Pregunta 2}

Un hormigón autocompactante bien elaborado se reconoce visualmente porque combina de manera adecuada fluidez, homogeneidad y cohesión, tres requisitos fundamentales descritos por el ACI 237R-07 para considerar que una mezcla se comporta realmente como HAC. Al observar el hormigón fresco, lo primero que debe apreciarse es una fluidez suficiente para que la mezcla se deforme y se expanda bajo su propio peso sin necesidad de vibrado. En el ensayo de \textit{slump flow}, por ejemplo, un HAC correctamente diseñado forma un círculo uniforme, con bordes relativamente limpios y sin interrupciones bruscas en su movimiento, alcanzando diámetros típicos entre 550 y 700 mm según la aplicación. Esta fluidez debe manifestarse de manera continua y estable, pero sin llegar a un estado acuoso que indique exceso de agua o pérdida de viscosidad.

En términos de homogeneidad, un HAC bien proporcionado presenta una distribución pareja de áridos finos y gruesos tanto en el centro como en los bordes del flujo. No deberían observarse acumulaciones localizadas de árido grueso ni zonas claras dominadas solo por pasta. La mezcla debe mantener un color uniforme y una textura consistente, lo que evidencia que la viscosidad es la adecuada para permitir el movimiento de todos los componentes sin separación. Esta combinación entre un bajo esfuerzo de cedencia y una viscosidad suficiente es la que garantiza que la mezcla fluya de forma conjunta durante su colocación.

Respecto de la cohesión y la estabilidad, un hormigón autocompactante correctamente diseñado no debería presentar segregación ni durante el flujo ni luego del asentamiento. Esto significa que no debe aparecer agua en la superficie (sangrado), ni deben generarse “islas’’ de árido grueso, ni separación evidente entre pasta y áridos. Visualmente, un HAC estable mantiene un borde definido, sin halos muy líquidos ni diferencias marcadas entre zonas de la mezcla. Ensayos simples como el \textit{Visual Stability Index} (VSI) ayudan a verificar esta condición, indicando valores aceptables entre 0 y 1 cuando la mezcla presenta buena cohesión y ausencia de segregación perceptible, según lo señalado por el ACI 237R-07.

Cuando el hormigón fresco no está adecuadamente preparado, estas características cambian de manera evidente. La segregación, uno de los defectos más comunes, se manifiesta como acumulación de árido grueso en los bordes del \textit{slump flow} o como zonas centrales dominadas solo por pasta, lo que indica un desequilibrio entre fluidez y viscosidad. Otra señal de mala preparación es la baja fluidez, observable cuando la mezcla no se expande lo suficiente, presenta bordes irregulares o se fractura durante el flujo; en estos casos, el \textit{slump flow} queda por debajo de los valores recomendados, sugiriendo un esfuerzo de cedencia demasiado alto. También puede evidenciarse falta de capacidad de paso cuando el hormigón es incapaz de atravesar espacios reducidos entre barras, algo que suele verificarse mediante ensayos como \textit{J-Ring} o \textit{L-Box}, donde diferencias importantes entre el flujo libre y el restringido, o relaciones $h_2/h_1$ menores a 0{,}8, indican riesgo de bloqueo.

Finalmente, ensayos sencillos como el \textit{slump flow}, el VSI, el \textit{J-Ring}, el \textit{L-Box} y la columna de segregación, todos incluidos en normas internacionales como ASTM y referidos por el ACI 237R-07, permiten identificar rápidamente estas deficiencias tanto en obra como en laboratorio. En conjunto, las observaciones visuales y estos ensayos permiten evaluar si un hormigón autocompactante ha sido correctamente diseñado, mezclado y manejado.


\subsection*{Pregunta 3}
El hormigón autocompactante (HAC) presenta un costo directo por metro cúbico superior al hormigón tradicional debido a su mayor contenido de finos, aditivos de hiper plastificantes y mayores exigencias de control durante la producción. Sin embargo, este mayor costo inicial suele compensarse mediante reducciones significativas en los costos indirectos y en los tiempos de ejecución en obra. El hormigón tradicional requiere mano de obra especializada para la vibración, un mayor número de operarios y equipos, así como ciclos de trabajo más largos para asegurar una compactación adecuada. En cambio, el HAC elimina la necesidad de vibrado, lo que reduce costos de operación, tiempos de colocación y desgaste de equipos. Además, su excelente fluidez mejora la terminación superficial y reduce reparaciones posteriores asociadas a oquedades, nidos de grava o segregación.

En términos de productividad, el HAC permite un vertido más rápido y continuo, disminuye cuellos de botella asociados a la vibración y reduce la fatiga del personal, aumentando la seguridad en obra. También mejora la sostenibilidad al disminuir el ruido, el consumo energético de equipos de vibrado y la probabilidad de retrabajos. Por el contrario, el hormigón tradicional, aunque más económico en materiales, presenta mayores riesgos de fallas de compactación, requiere más personal en faenas congestionadas y tiene un impacto operacional mayor cuando se ejecutan elementos complejos.

\subsubsection*{Situaciones donde el HAC es más ventajoso}

El uso de HAC es especialmente recomendable en elementos con alta congestión de armaduras, como muros estructurales delgados, pilares confinados y zonas de acoplamiento en edificios en altura, donde el hormigón tradicional presenta alto riesgo de segregación o compactación deficiente. También es ventajoso en elementos arquitectónicos o estructurales donde se requieren altas exigencias de acabado superficial, tales como muros a la vista, placas delgadas o elementos prefabricados con geometrías irregulares, debido a su excelente capacidad de llenado y mínima necesidad de retrabajo.

\subsubsection*{Situación donde el HAC no es recomendable}

El uso de HAC no resulta económicamente justificable en elementos masivos, de baja complejidad geométrica y con armaduras poco congestionadas, como fundaciones corridas, bloques masivos o losas de baja exigencia. En estos casos, el costo extra del HAC no genera ganancias significativas en productividad ni en calidad final, ya que el hormigón tradicional puede compactarse adecuadamente con vibrado convencional y a un costo mucho menor por metro cúbico.


\subsection*{Pregunta 4}

En la elaboración del hormigón autocompactante, los aditivos químicos cumplen distintas funciones, como el que permiten obtener simultáneamente una alta fluidez, una adecuada cohesión y una estabilidad suficiente para evitar la segregación. Los aditivos más utilizados son los superplastificantes de última generación basados en policarboxilatos, los modificadores de viscosidad (VMA), los reductores de agua de rango medio y, en ciertos casos, los aditivos inclusores de aire cuando las condiciones ambientales requieren resistencia frente a ciclos de congelación y deshielo. El superplastificante policarboxílico es el componente muy importante del HAC, ya que permite alcanzar una elevada capacidad de deformación con una baja relación agua/cementante. A nivel físico y químico actúa por medio de la adsorción de sus moléculas sobre las partículas de cemento, generando repulsión electrostática y, especialmente, dispersión estérica. La combinación de estos mecanismos libera agua atrapada en los flóculos, reduce de manera significativa el esfuerzo de cedencia y mejora la fluidez sin separar los áridos de la pasta. Cuando la dosificación es insuficiente, el hormigón pierde fluidez, no alcanza un \textit{slump flow} característico de un HAC y puede presentar bloqueo entre armaduras; cuando la dosificación es excesiva, la mezcla puede volverse inestable, con tendencia a la segregación, el sangrado y la pérdida de cohesión debido a una viscosidad demasiado baja.

Los modificadores de viscosidad, por su parte, permiten estabilizar la mezcla cuando se trabaja con niveles altos de fluidez. Muchos de ellos están basados en polímeros derivados de celulosa o goma xantana, que forman una red coloidal capaz de retener agua y aumentar la viscosidad plástica sin afectar en exceso el esfuerzo de cedencia. Su acción controla la movilidad interna de la pasta, reduce el riesgo de segregación dinámica durante el flujo y mantiene la mezcla homogénea una vez en reposo. Una dosificación insuficiente de VMA puede provocar segregación visible, exudación superficial y falta de uniformidad en el desplazamiento; en cambio, una dosificación excesiva puede generar un hormigón demasiado espeso, con pérdida de capacidad de autocompactación y comportamiento similar al de un hormigón fluido convencional.

Los reductores de agua de rango medio pueden utilizarse como complemento para ajustar la reología en mezclas con características particulares de áridos o finos. Aunque su efecto es menor que el de un superplastificante policarboxílico, contribuyen a disminuir la demanda de agua y a mejorar la manejabilidad inicial. Finalmente, los inclusores de aire se emplean únicamente cuando la estructura requiere resistencia frente a condiciones de congelamiento y deshielo. Estos aditivos generan microburbujas estables distribuidas en la pasta, pero su uso en un HAC debe ser cuidadosamente controlado, ya que la interacción con los superplastificantes puede alterar la estabilidad del sistema de aire. Un exceso de aire reduce la resistencia mecánica, mientras que un defecto hace al hormigón vulnerable al deterioro por congelamiento y deshielo.

\subsection*{Pregunta 5}

Los morteros de reparación de MC Bauchemie están formulados para restituir secciones dañadas de hormigón. Se caracterizan por su alta adherencia al sustrato, baja retracción, buena trabajabilidad y compatibilidad mecánica con el hormigón existente. Se aplican manualmente o proyectados en capas, permitiendo reconstruir geometrías y recuperar capacidad estructural.

El grout es un material fluido o semifluido, sin retracción y de alta capacidad de relleno, diseñado para ocupar vacíos confinados y asegurar una correcta transmisión de cargas. Presenta altas resistencias iniciales, buena fluidez y excelente comportamiento en rellenos bajo placas base, anclajes o maquinaria.

\subsubsection*{Funciones y propiedades principales}

Los morteros de reparación restauran superficies deterioradas, recubrimientos de armaduras y desprendimientos localizados, trabajando bien en vertical y sobre cabeza. En cambio, los grout se utilizan donde se requiere un flujo continuo y completo del material, especialmente en zonas donde la compactación mecánica no es posible.

\subsubsection*{Ejemplos de uso en obra}

Se recomienda mortero de reparación para reparar bordes de losas, columnas dañadas, recubrimientos corroídos o superficies con desprendimientos. El grout es más adecuado para rellenar placas base de estructuras metálicas, nivelar y anclar equipos, fijar pernos, rellenar pedestales o vacíos estrechos.

\subsection*{Pregunta 6}

En las regiones del sur de Chile, donde la disponibilidad de áridos proviene principalmente de depósitos fluviales monotamaño y con textura superficial pulida, es necesario ajustar el diseño de mezclas de hormigón para asegurar la reología, la estabilidad y la bombeabilidad adecuadas, especialmente cuando se produce hormigón autocompactante. Los áridos fluviales suelen presentar granulometrías con escasa fracción fina y con curvas muy uniformes, lo que genera un esqueleto granular con mayor cantidad de vacíos y una menor capacidad de trabamiento mecánico entre partículas. Esta condición reduce la cohesión del hormigón, aumenta el riesgo de segregación y disminuye la viscosidad plástica de la mezcla, lo que afecta directamente la estabilidad dinámica y estática del hormigón fresco. Además, la textura pulida y el carácter redondeado de estos áridos disminuye la fricción interna, favoreciendo el flujo pero perjudicando la estabilidad, especialmente en mezclas de alta fluidez como el HAC, donde la ausencia de finos y la falta de angularidad pueden provocar separación de áridos durante el transporte o dentro de la tubería de bombeo.

Estas particularidades locales obligan a incrementar el contenido de finos mediante mayor cantidad de cemento, incorporación de filler calizo o uso de adiciones minerales como puzolanas o cenizas volantes, con el fin de densificar la pasta, mejorar el empaquetamiento y aumentar la cohesión. El aumento de finos contribuye a elevar la viscosidad plástica y a reducir la susceptibilidad a la segregación, lo que resulta importante para mantener la autocompactabilidad sin comprometer la estabilidad del hormigón. En este contexto, los aditivos superplastificantes desempeñan un rol fundamental al permitir alcanzar altos niveles de fluidez con una relación agua cementante controlada. Sin embargo, debido a la alta variabilidad de humedad en los acopios de áridos propia del clima lluvioso del sur, la dosificación de superplastificantes debe ajustarse constantemente, ya que cambios en la humedad modifican el agua efectiva y pueden provocar mezclas rígidas o excesivamente fluidas. Paralelamente, los modificadores de viscosidad resultan esenciales para estabilizar la mezcla, dado que forman una red coloidal capaz de retener agua y aumentar la viscosidad sin afectar significativamente la capacidad de deformación, permitiendo que el HAC mantenga su integridad bajo altos niveles de flujo.

El clima frío y húmedo propio del sur de Chile influye también en los tiempos de hidratación, ralentizando el fraguado y el desarrollo inicial de resistencias. En obras que requieren puesta en servicio rápida, prefabricados o desencofrados tempranos, puede ser necesaria la incorporación de acelerantes para contrarrestar estos efectos. Por otro lado, cuando las condiciones de transporte incluyen distancias prolongadas, caminos complejos o accesos rurales, resulta fundamental controlar la retención de asentamiento y prolongar el tiempo abierto de la mezcla. En estos casos, puede justificarse el uso de retardadores o de superplastificantes con liberación gradual, que permiten mantener la manejabilidad sin pérdida abrupta de consistencia durante el viaje hacia la obra, condición especialmente relevante para hormigones autocompactantes.

Al relacionar estas necesidades con la oferta de productos MC Bauchemie, se observa que los superplastificantes de la familia MC PowerFlow son adecuados para alcanzar altos niveles de fluidez y buena retención de asentamiento incluso bajo variaciones de humedad en los áridos. Para mejorar la cohesión y reducir el riesgo de segregación asociado a áridos fluviales monotamaño, los modificadores de viscosidad de la línea Centrament VMA proporcionan la estabilidad requerida sin afectar la autocompactación. En situaciones donde el aumento de finos es indispensable, los fillers y adiciones compatibles con los sistemas de aditivos MC permiten densificar la matriz y mejorar la estabilidad reológica del HAC. Para enfrentar temperaturas bajas o necesidades de rápido desarrollo de resistencias, los acelerantes Centrament Rapid resultan adecuados; mientras que para distancias prolongadas de transporte o condiciones logísticas complejas, los retardadores MC Retard permiten preservar la trabajabilidad durante períodos extendidos. Estas elecciones se justifican porque el sur de Chile combina áridos fluviales de baja estabilidad, clima lluvioso y frío, altos niveles de humedad, variabilidad en la dosificación de agua y obras ubicadas a grandes distancias de las plantas de hormigón, condiciones que exigen sistemas reológicos estables y aditivos capaces de adaptarse con precisión a variaciones ambientales y granulométricas.



\newpage
\section*{Referencias}
[1] ACI Committee 237. Self-Consolidating Concrete (ACI 237R-07). 
    American Concrete Institute, Farmington Hills, MI, 2007.
\end{document}
